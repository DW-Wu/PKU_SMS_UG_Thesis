% PKU Math undergradaute thesis template 2022 by Uwi Ewad
% inspired by Bochen Tan (2019): https://github.com/tbcdebug/PKU_EECS_UGR_THSS
% and Casper Ti. Vector (2017): https://github.com/JoeHF/aet_paper

\documentclass[UTF8,openany,a4paper]{ctexbook}
\usepackage{amsmath}
\usepackage{amsthm}
\usepackage{amssymb}
\usepackage{mathrsfs}
\usepackage{tikz-cd}
\usepackage{setspace}
\usepackage{fancyhdr}
\usepackage{booktabs}
\usepackage{inputenc}
\usepackage{hyperref}
\usepackage{bm,ulem}
\usepackage{geometry,graphicx,float}
\usepackage{xeCJK}
\usepackage{cite}
\usepackage{fontspec}
\usepackage{lipsum,zhlipsum}

\geometry{left=3.18cm,right=3.18cm,top=2.54cm,bottom=2.54cm}

\setCJKmainfont[BoldFont={STHeiti}, ItalicFont={STKaiti},BoldItalicFont={Kaiti SC Bold}]{STSong}
\setCJKsansfont{Yuanti SC}
\setCJKmonofont{STFangsong}

% \setmainfont{Times New Roman}

% some font shape causes warnings...
\DeclareFontFamily{U}{rsfs}{\skewchar\font127 }
\DeclareFontShape{U}{rsfs}{m}{n}{%
    <-6> rsfs5
    <6-8> rsfs7
    <8-> rsfs10
}{}
\DeclareFontShape{TU}{lmr}{m}{scit}{
    <-> qx-lmcsco10
}{}

% set format of titles
\ctexset{
    chapter={format={\centering\zihao{3}\bfseries}, beforeskip=0pt},
    section={format={\zihao{3}\rmfamily},beforeskip=0pt},
    subsection={format={\zihao{4}\rmfamily},beforeskip=0pt}
}

\newtheorem{theorem}{定理}
\newtheorem{proposition}[theorem]{命题}
\newtheorem{lemma}[theorem]{引理}
\newtheorem{corollary}[theorem]{推论}
\theoremstyle{definition}
\newtheorem{definition}{定义}
\newtheorem{remark}[theorem]{注}

\title{Fisher信息矩阵在多层感知机中的各种应用}
\author{Luster Dawn}

% get title
\makeatletter
\let\Title\@title
\makeatother

\begin{document}
% header and footer
\fancypagestyle{pkuthesis}{
    \ctexset{chapter/pagestyle=pkuthesis}
    \fancyhead[L,R]{}
    \fancyhead[C]{\Title}
    \fancyfoot[C]{第 \thepage 页}
}
\pagestyle{pkuthesis}


% the metrics are approximately observed from the 2022 template MS Word file
\pagenumbering{roman}
\begin{titlepage}
    \centering
    \includegraphics[width=25mm]{pkulogo.eps}
    \hspace{3mm}
    \includegraphics[width=72mm]{pkuword.eps}

    \vspace{0.5cm}

    {\zihao{0}\bfseries 本科生毕业论文}

    \vspace{1cm}

    \begin{tabular}{rc}
        \rmfamily\zihao{2}题目:
        & \parbox[b][2.2cm]{10.16cm}{\centering\zihao{2}\bfseries \Title} \\ \cline{2-2} 
        & \parbox[c][2.2cm]{10.16cm}{\centering\zihao{2}\bfseries The Various Applications of FIM in MLP} \\ \cline{2-2} 
    \end{tabular}

    \vspace{3.5cm} % adjust the vertical lengths if the title was too long

    \begin{tabular}{rc}
        \parbox[b][0.89cm]{3.19cm}{\bfseries\zihao{3}姓\phantom{某某}名:}
        & \parbox[b]{7.62cm}{\centering\ttfamily\zihao{3} Sheen Daybreak} \\ \cline{2-2}
        \parbox[b][0.89cm]{3.19cm}{\bfseries\zihao{3}学\phantom{某某}号:}
        & \makebox[7.62cm]{\centering\rmfamily\zihao{3}20101010} \\ \cline{2-2}
        \parbox[b][0.89cm]{3.19cm}{\bfseries\zihao{3}院\phantom{某某}系:}
        & \makebox[7.62cm]{\centering\ttfamily\zihao{3}魔法学院} \\ \cline{2-2}
        \parbox[b][0.89cm]{3.19cm}{\bfseries\zihao{3}专\phantom{某某}业:}
        & \makebox[7.62cm]{\centering\ttfamily\zihao{3}炼丹学} \\ \cline{2-2}
        \parbox[b][0.89cm]{3.19cm}{\bfseries\zihao{3}导师姓名:}
        & \makebox[7.62cm]{\centering\ttfamily\zihao{3}Dusk Glitter} \\ \cline{2-2} 
    \end{tabular}
    
    \vspace{3cm}

    {\rmfamily\zihao{3}二〇二二年六月}
\end{titlepage}

% line space of text
\doublespacing

\chapter*{版权声明}
任何收存和保管本论文各种版本的单位和个人,未经本论文作者同意,不得将本论文转借他人,亦不得随意复制、抄录、拍照或以任何方式传播。否则,引起有碍作者著作权之问题,将可能承担法律责任。

\chapter*{摘要}
\zhlipsum[5-10][name=zhufu]

\chapter*{ABSTRACT}
\lipsum[1-5][1-10]

\tableofcontents

\chapter{一只敏捷的棕色狐狸}
\setcounter{page}{1} % start counting pages here
\pagenumbering{arabic}
\textit{This is italic text and \textsc{slanted SC text}}.
\section{敏捷}
\zhlipsum[1-2]
\begin{theorem}
    \zhlipsum[1][name=zhufu]
    所以FIM \footnote{脚注: FIM含义为Fisher information matrix} ……
\end{theorem}
\subsection{一只}
\zhlipsum[3]
\subsection{的}
\zhlipsum[4-5]
\section{棕色}
\zhlipsum[6]
\subsection{狐狸}
\zhlipsum[7]

\chapter{跳过一只懒狗}
\zhlipsum[9]
\section{跳过一只懒狗}
\subsection{跳过}
\zhlipsum[8]
\subsection{懒狗}
\zhlipsum[10]

{\zihao{4} \sffamily 一只敏捷的棕色狐狸。} \cite{bax2005}
\zhlipsum[11]

{\zihao{5} \ttfamily 一只敏捷的棕色狐狸。} \cite{doumic2009}
\zhlipsum[12]

\bibliographystyle{amsplain}
\bibliography{citing.bib}

\appendix
\chapter{定理证明}
\begin{lemma}
    \zhlipsum[13]    
\end{lemma}
\begin{lemma}
    \zhlipsum[14]    
\end{lemma}
\begin{lemma}
    \zhlipsum[15]    
\end{lemma}
\chapter*{致谢}
\zhlipsum[16]

\chapter*{北京大学学位论文原创性声明和使用授权说明}
\begin{center}
\zihao{3}\rmfamily 原创性声明
\end{center}

本人郑重声明:所呈交的学位论文,是本人在导师的指导下,独立进行研究工作所取得的成果。除文中已经注明引用的内容外,本论文不含任何其他个人或集体已经发表或撰写过的作品或成果。对本文的研究做出重要贡献的个人和集体,均已在文中以明确方式标明。本声明的法律结果由本人承担。
\bigskip

\begin{flushright}
论文作者签名: \makebox[37mm]{}

日期:\qquad 年\qquad 月\qquad 日
\end{flushright}
\vspace{100pt}

\begin{center}
\zihao{3}\rmfamily 学位论文使用授权说明
\end{center}

\bigskip

本人完全了解北京大学关于收集、保存、使用学位论文的规定,即:
\begin{itemize}
    \item 按照学校要求提交学位论文的印刷本和电子版本;
    \item 学校有权保存学位论文的印刷本和电子版,并提供目录检索与阅览服务,在校园网上提供服务;
    \item 学校可以采用影印、缩印、数字化或其它复制手段保存论文。
\end{itemize}
\bigskip
\begin{flushright}
    论文作者签名: \makebox[20mm]{}
    导师签名:\makebox[20mm]{}
        
    日期:\qquad 年\qquad 月\qquad 日
\end{flushright}
\end{document}